\documentclass[12pt]{article}
\usepackage[russian]{babel}
\usepackage{color}
\begin{document}
{\Large \bfseries EquationEditor} – это средство визуального редактирования, предоставляющее набор \underline{стандартных математических конструкций}, которые вы можете заполнять числами, специальными символами и другими структурными частями формул. Чтобы отредактировать одну из формул, созданных с помощью {\it EquationEditor}, достаточно дважды щелкнуть по ней или, выделив формулу, выбрать команду {\bfseries \textcolor{blue}{Правка ? Объект Equation}}, а из появившегося подменю выбрать пункт {\bfseries \textcolor{blue}{Изменить}}. Это приведет к запуску {\it EquationEditor} и вставке в него выбранной формулы для редактирования.


\underline{Примеры математических формул}, приведены в этой главе, изображены так, как вы видите их на экране. 
{\it EquationEditor} предоставляет немало мощных средств для настройки внешнего вида и процесса набора формул. В то же время стандартные настройки и стили {\it EquationEditor} подходят для большинства математических, научных и деловых  работ.
\begin{center}


Создание формулы напоминает сборку {\sl трехмерной головоломки}: соединяя составные части по одной, вы стремитесь создать завершенную форму, к примеру, шар или куб. Если одна из составных частей установлена {\bf \textcolor{red}{неверно}}, то конечного результата вам достигнуть {\bfseries \textcolor{red}{не удастся}}.
\end{center}



\begin{itemize}
\item Adobe Photoshop
\item Adobe Premier
\item Adobe After Effects
\end{itemize}

\renewcommand{\theenumi}{(\asbuk{enumi})}
\renewcommand{\labelenumi}{\asbuk{enumi})}
\definecolor{light-blue}{rgb}{0.8,0.85,1}
\colorbox{light-blue}{\bf Главные правила студента:}
\begin{enumerate}
\item\label{boss} не прогуливать;
\item вовремя выполнять домашние задания;
\item не мешать преподавателям вести пары.
\end{enumerate}
\end{document}